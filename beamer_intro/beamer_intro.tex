%Choosing the class, here you can also change font and such
\documentclass{beamer}

%Setting up the theme
\usetheme{metropolis}

%Set the style for blocks
\setbeamertemplate{blocks}[rounded][shadow=true]


\usepackage[utf8]{inputenc}

%Some additional hyperref options to make the refs look good
\usepackage{hyperref}
\hypersetup{
    colorlinks=true,
    linkcolor=blue,
    filecolor=magenta,      
    urlcolor=cyan,
}
\urlstyle{same}


%Package and settings to display Latex code
%There are other options for displaying code worth checking out
\usepackage{listings}
%listings ist nicht utf8 kompatibel. Deshalb sind diese Definitionen notwendig.
\lstset{literate=%
  {Ö}{{{\color{magenta}\"O}}}1
  {Ä}{{{\color{magenta}\"A}}}1
  {Ü}{{{\color{magenta}\"U}}}1
  {ß}{{{\color{magenta}\ss}}}1
  {ü}{{{\color{magenta}\"u}}}1
  {ä}{{{\color{magenta}\"a}}}1
  {ö}{{{\color{magenta}\"o}}}1
}
\lstset{
    language=[LaTeX]TeX,
    breaklines=true,
    basicstyle=\tt\scriptsize,
    keywordstyle=\color{blue},
    identifierstyle=\color{magenta},
}

%Quellenangaben
%Die backref option gibt an auf welcher Seite das Material zitiert wurde
\usepackage[backend=biber, backref=true]{biblatex}
\bibliography{bibliography.bib}

%Information to be included in the title page:
\title{Beamer - Präsentationen mit \LaTeX}
\author{Christian Kirfel}
\institute{Physikalisches Institut Bonn}
\date{2020}



\begin{document}

\frame{\titlepage}


\begin{frame}{Was ist eigentlich \LaTeX Beamer?}
    \begin{center}
        Beamer ist eine Klasse des Textsatzprogramms LaTeX, mit der Präsentationen erstellt werden können.
        Beamer wurde von Till Tantau entwickelt und erstmals 2003 auf dem CTAN-Netzwerk veröffentlicht.
        Seit April 2010 wird das Paket von Vedran Miletić gepflegt. Weiterhin lieferte Joseph Wright Beiträge zu dem Projekt.
        Der Name der Klasse ist vom verbreiteten deutschen Scheinanglizismus für einen Videoprojektor abgeleitet, der zum Vorführen einer Präsentation verwendet wird.\cite{wiki:beamer}
    \end{center}
\end{frame}

\begin{frame}{Warum sollte man \LaTeX Beamer nutzen?}
    \begin{block}{Vorteile}
        \begin{itemize}
            \item alle Vorteile von \LaTeX
                \begin{itemize}
                    \item Konsistente Handhabung von Referenzen und Zitaten
                    \item Trennung von Inhalt und Formattierung 
                    \item Einfache Formatierung von mathematischen Formeln und Symbolen
                \end{itemize}
            \item leichte Wiederverwandbarkeit von alten Präsentationen
            \item automatische Folienaufteilung
            \item Verwendung von Formeln und Text direkt auf \LaTeX Dokumenten
        \end{itemize}
    \end{block}
    \begin{block}{Nachteile}
        \begin{itemize}
            \item Ineffizient für schnelle Arbeiten
            \item Mühsam bei der genauen Anordnung von Elementen
        \end{itemize}
    \end{block}
\end{frame}

% Setting up the Beamer package

\begin{frame}[containsverbatim]{Der Beginn eines \LaTeX Beamer Projektes}
    \begin{lstlisting}
        \documentclass{beamer}

        \usepackage[utf8]{inputenc}

        \begin{document}
    \end{lstlisting}
\end{frame}

% Eine einfache Folie

\begin{frame}[containsverbatim]{Eine einfache Folie}
    \begin{lstlisting}
    \begin{frame}{Titel der Folie}
        \begin{center}
        \Huge EingroßeszentriertesWort
        \end{center}
    \end{frame}
    \end{lstlisting}
\end{frame}

\begin{frame}{Titel der Folie}
    \begin{center}
        \Huge EingroßeszentriertesWort
    \end{center}
\end{frame}

% Die Titel Folie

\begin{frame}[containsverbatim, fragile]{Die Titel Folie}
    \begin{lstlisting}
        \begin{document}

        \usetheme{metropolis}

        \title{Beamer - Präsentationen mit \LaTeX}
        \author{Christian Kirfel}
        \institute{Physikalisches Institut Bonn}
        \date{2020}

        \frame{\titlepage}
    \end{lstlisting}
\end{frame}

%Display a title frame
\frame{\titlepage}

%Some tips and tricks

\begin{frame}[containsverbatim]{Bilder und Grafiken}
    \begin{itemize}
        \item die figure Umgebung kann in frames verwendet werden.
    \end{itemize}
    \begin{lstlisting}
    \begin{frame}{Titel der Folie}
        \begin{figure}
            \centering
            \includegraphics[width = 0.5\textwidth]{machine.eps}
            \caption{Eine caption}
            \label{fig:einLabel}
        \end{figure}
    \end{frame}
    \end{lstlisting}
\end{frame}

\begin{frame}{Titel der Folie}
    \begin{figure}
        \centering
        \includegraphics[width = 0.5\textwidth]{machine.eps}
        \caption{Eine caption}
        \label{fig:einLabel}
    \end{figure}
\end{frame}

\begin{frame}[containsverbatim]{Eine Folie aufteilen - 1}
    \begin{itemize}
        \item Eine Möglichkeit ist die Nutzung von columns
    \end{itemize}
    \begin{lstlisting}
        \begin{frame}{Titel der Folie}
            \begin{columns}
                \begin{column}{0.5\textwidth}
                \end{column}
                \begin{column}{0.5\textwidth}
                \end{column}
            \end{columns}
        \end{frame}
    \end{lstlisting}
\end{frame}


\begin{frame}{Titel der Folie}
    \begin{columns}
        \begin{column}{0.5\textwidth}
            \begin{itemize}
                \item Eins
                \item Zwei
                \item Drei
            \end{itemize}
        \end{column}
        \begin{column}{0.5\textwidth}
            \begin{itemize}
                \item Eins
                \item Zwei
                \item Drei
            \end{itemize}
        \end{column}
    \end{columns}
\end{frame}

%Achtung! In dieser Folie ist das listing nicht ganz abgebildet aber das code example ist auf der naechsten Folie

\begin{frame}[containsverbatim]{Eine Folie aufteilen - 2}
    \begin{itemize}
        \item Alternativ kann man eine tabular Umgebung nutzen
    \end{itemize}
    \begin{columns}
        \begin{column}{0.5\textwidth}
            \begin{lstlisting}[basicstyle=\fontsize{8}{9}\selectfont]
            \begin{tabular}{p{5cm}|p{5cm}}
            \begin{figure}
                \includegraphics[scale = 0.09]{brain}
            \end{figure}
            & 
            \begin{figure}
                \includegraphics[scale = 1.4]{machine}
            \end{figure} \\
          \multicolumn{1}{c|}{Humam senses} & \multicolumn{1}{c}{Input variables} \\
            \begin{itemize}
                \item Extraction of relevant info
            \end{itemize}
            & 
            \begin{itemize}
              \item Preprocessed by user
            \end{itemize} \\
            \end{lstlisting}
        \end{column}    
        \begin{column}{0.5\textwidth}
            \begin{lstlisting}[basicstyle=\fontsize{8}{9}\selectfont]
                    \multicolumn{1}{c|}{Human brain} & \multicolumn{1}{c}{Net of nodes} \\
            \begin{itemize}
                \item Single combination $\rightarrow$ action
            \end{itemize}
            & 
            \begin{itemize}
              \item Combination forms non-linear model
            \end{itemize} 
         \end{tabular}
        \end{frame}
            \end{lstlisting}
        \end{column}    
    \end{columns}
\end{frame}


\begin{frame}{A tabular slide}
    \begin{tabular}{p{5cm}|p{5cm}}
        \begin{figure}
            \includegraphics[scale = 0.09]{brain}
        \end{figure}
        & 
        \begin{figure}
            \includegraphics[scale = 1.4]{machine}
        \end{figure} \\
      \multicolumn{1}{c|}{Humam senses} & \multicolumn{1}{c}{Input variables} \\
        \begin{itemize}
            \item Extraction of relevant info
            \item Impossible for machines
        \end{itemize}
        & 
        \begin{itemize}
          \item Preprocessed by user
          \item {e.g.} kinematic variables
        \end{itemize} \\
    \multicolumn{1}{c|}{Human brain} & \multicolumn{1}{c}{Net of nodes} \\
        \begin{itemize}
            \item Web of neuron cells
            \item Input from surrounding cells
            \item Single combination $\rightarrow$ action
        \end{itemize}
        & 
        \begin{itemize}
          \item Nodes = simple processors
          \item Connected by linear function
          \item Combination forms non-linear model
        \end{itemize} 
     \end{tabular}
    \end{frame}


\begin{frame}[containsverbatim]{Eine Folie aufteilen - 3}
    \begin{itemize}
        \item Für noch kompliziertere Aufteilungen kann man minipages oder subframes nutzen.
    \end{itemize}
\end{frame}



\begin{frame}[containsverbatim]{Folien aufbauen}
    \begin{lstlisting}
    \begin{frame}{Eine Folie mit Pausen}
        \begin{itemize}
            \item Eins \pause
            \item Zwei \pause
            \item Drei \pause
        \end{itemize}
    \end{frame}
    \end{lstlisting}
\end{frame}

\begin{frame}{Eine Folie mit Pausen}
    \begin{itemize}
        \item Eins \pause
        \item Zwei \pause
        \item Drei 
    \end{itemize}
\end{frame}

\begin{frame}[containsverbatim]{Blöcke}
    \begin{lstlisting}
        \setbeamertemplate{blocks}[rounded][shadow=true]
        \begin{frame}{Verschiedene Blöcke}
            \begin{block}{Ein Block}
                Text im Block
            \end{block}
            \begin{alertblock}{Ein Alert Block}
                Text im Block
            \end{alertblock}
            \begin{definition}
                $E = mc^2$
            \end{definition}
        \end{frame}
    \end{lstlisting}
\end{frame}

\begin{frame}{Verschiedene Blöcke}
    \setbeamertemplate{blocks}[rounded][shadow=true]
    \begin{block}{Ein Block}
        Text im Block
    \end{block}
    \begin{alertblock}{Ein Alert Block}
        Text im Block
    \end{alertblock}
    \begin{definition}
        $E = mc^2$
    \end{definition}
\end{frame}

\begin{frame}[containsverbatim]{Quellenangaben}
    \begin{lstlisting}
        \usepackage[backend=biber]{biblatex}
        \bibliography{bibliography.bib}
        \begin{frame}{Quellen}
            \printbibliography
        \end{frame}
    \end{lstlisting}
\end{frame}

\begin{frame}{Fazit}
    \begin{itemize}
        \item Fast alle packages und Möglichkeiten von \LaTeX lassen sich weiter anwenden
        \item Refs, hyperrefs und bibliography funktinieren wie gehabt
        \item Für alle Stilelemente gibt es unbegrenzte Möglichkeiten und templates zu testen
        \item \href{https://www.overleaf.com/learn/latex/Beamer}{Beamer tutorial}
    \end{itemize}
\end{frame}

\begin{frame}{Quellen}
    \printbibliography
\end{frame}


\end{document}